\documentclass[12pt,a4paper,notitlepage]{article}
\usepackage[utf8]{inputenc}
\usepackage{amsmath}
\usepackage{amsfonts}
\usepackage{amssymb}
\usepackage[T1]{fontenc}
\usepackage[a4paper, left=.6in,right=.6in,top=.8in,bottom=.8in,]{geometry}
\usepackage{tabularx,ragged2e,booktabs,caption}
\usepackage{setspace}
\setstretch{1.5}

\begin{document}
\title{The futility of mercantilist wars. A case study of France and Hamburg between 1713 and 1820}
\author{
  Guillaume Daudin \\ Université Paris-Dauphine \\guillaume.daudin@dauphine.fr		
  \and
  Elisa Tirindelli \\ PhD at Trinity College Dublin  \\ tirindee@tcd.ie
}
\maketitle

\section*{Abstract}
Our paper examines whether mercantilist warfare was or not effective in its own terms, by crippling trade of the defeated power. We do so by looking at product-level bilateral French  trade data from 1713 to 1821. Atlantic trade was obviously affected by maritime wars between France and the United Kingdom. Did war lead to long-term reduction of French trade? How was non-Atlantic trade affected?
The existing literature looking at the impact of wars on neutral trade comes to mixed results. Barbieri and Levy (1999) for example, analyse the impact of war from 1870 to 1992, and find that the general impact of conflict on trade is not particularly strong and mostly only temporary. Anderton and Carter (2001), on the other hand, look at the effect of wars on global trade, and find that when major world power are at war significant pre and post war effects are observed. Finally, Rahman (2007), using British trade data from eighteenth century, finds that it is warfare between naval powers that brings disruption to trade. 
To our belief we are the first ones to look at the effect of wars on good-specific bilateral trade flows. We are looking in particular at coffee and sugar, which are the major colonial goods, and at wine and eau de vie, major European products. We focus, on the particular case of Hamburg, for two reasons. First, it offers import trade statistics that allow double checking the French data. Second, it was an important trade partner, a neutral gateway to Germany whose trade was mainly directly affected by war. We then repeat the experiment on all French trading partners as an aggregate. We also explore whether the effect of war depends on the identity of the victor and the belligerent status of trade partners.
We find a general negative impact of war on French exports to Hamburg and all aggregate trading partners, yet, we observe big differences depending on the products. Wars have a very large and negative impact on colonial products, but a positive one on wine (and a very positive one on eau-de-vie). This suggests that trade of some specific products was benefitting from wars. Furthermore, we find little long-term effect of wars before 1793. It was not possible for the United Kingdom to cripple French trade before the ideological wars that started after the Revolution. 

\textbf{Keywords}: international trade, 18th century, France, Hamburg, warfare
\end{document}