\documentclass[12pt,a4paper,titlepage,english]{article}
\usepackage[utf8]{inputenc}
\usepackage{amsmath}
\usepackage{amsfonts}
\usepackage{longtable}
\usepackage{amssymb}
\usepackage{graphicx}
\usepackage[a4paper, left=.6in,right=.6in,top=.8in,bottom=.8in,]{geometry}
\usepackage{tabularx,ragged2e,booktabs,caption}
\usepackage{setspace}
\setstretch{1.5}
\usepackage{tabularx, booktabs}
\usepackage{dcolumn} 
  \newcolumntype{d}[1]{D{.}{.}{#1}}    
\newcolumntype{Y}{>{\centering\arraybackslash}X}
\usepackage[T1]{fontenc}
\usepackage{babel}
\usepackage{epigraph}
\usepackage{natbib}

%\newcommand{\source}[1]{\caption*{\footnotesize Source: {#1}} }
\usepackage{atbegshi}% http://ctan.org/pkg/atbegshi
\AtBeginDocument{\AtBeginShipoutNext{\AtBeginShipoutDiscard}}



\author{
  Guillaume Daudin \\ Université Paris-Dauphine
  \and
  Elisa Tirindelli \\ Trinity College Dublin 
}
\title{The futility of mercantilist wars \\ Anglo-French conflicts during the eighteenth century}

\begin{document}


\thanks{The authors want to thanks Philip Hoffman for sharing data with them.}

\maketitle
Is mercantilist warfare effective in its own terms, by crippling trade of defeated powers? Our paper explores the Anglo-French experience during the eighteenth century and contributes to understanding why that was not the case.
\cite{jefferson_letter_1823} famously noticed that European nations « were nations of eternal war». Indeed, from 1700 to 1825, 2 years out of 3 experienced conflict between major European powers \cite{roser_war_2016}. Rivalry between Great-Britain and France was central, so much as the period between 1688 to 1815 was called the « 2nd Hundred Years War » 1688-1815. War has many causes Yet, especially after the death of Louis XIV, it cannot be denied that mercantile rivalry was an important motivation of Anglo-French wars (\cite{wallerstein_modern_1980, crouzet_guerre_2008}). Each nation was jealous of the other’s commercial success. The British believed war was a good way to curtail them. The French partly agreed and were more wary of wars because they did not have much naval success. 
Here is the long list of wars between France and Britain after the death of Louis XIV : War of the Polish Succession (1733-1738) (little naval hostilities), War of the Austrian Succession (1740 (naval hostilities started in 1744)–1748), Seven Years' War (1756–1763), War of American independence (1775 (French involvement started in 1778)–1783), French Revolutionary Wars (1792–1802) and Napoleonic Wars (1803–1815). Yet, all these wars were in vain before the 1790s, as French trade increased up to the British level throughout the eighteenth century (Figure \ref{FrBritTrade}).
Looking at peace-time trends (including land only wars), it is clear that French trade, despite big war shocks, was resilient and was not moved out of its pre-1744 trend (Figure \ref{FrPeaceTrade}). Things changed after 1815.

\begin{figure}
\caption{French, British trade and Anglo-French wars}
\centering
\includegraphics[scale=.6]{"Total silver trade FR GB".png}
%\source{French trade up to 1821: \cite{daudin_toflit18_????}. French trade 1822-1840: \cite{federico_world_2016} / \cite{dedinger_exploring_2017},

England/British trade up to 1800: \cite{deane_british_1969}. UK trade from 1801 to 1840: \cite{federico_world_2016} / \cite{dedinger_exploring_2017},

Livre tournois silver value: \cite{de_wailly_memoire_1857} and \cite{hoffman_priceless_2000}; Pound sterling silver value: \cite{clark_england_1209-1914_2006} and \cite{jastram_silver_1981}}
\label{FrBritTrade}
\end{figure}




\begin{figure}
\caption{Peace time trends of total French trade}
\centering
\includegraphics[scale=.6]{"Peace-time trends of French trade".png}

\source{see Figure \ref{FrBritTrade} and author's computations}
\label{FrPeaceTrade}
\end{figure}

How come the pre-1792 wars did not have a lasting effect on French trade? This is important to understand the effect of wars in general, the geopolitical history of the eighteenth and nineteenth century and the globalization/deglobalization cycle from the 1490s to the 1840s.

There exists a vast literature focusing on the relationship between trade and war.
A first strand of this literature concentrates on the impact of trade on wars. Within this strand, two major perspectives have emerged: a liberal and a realist one. The first supports a vision of interdependence between trade and war, pointing out that trade promotes peace since it is a better method of expansion than wars. The second opposes this view by claiming that there is no impact of trade on wars, and if any, then it will be a positive impact, as countries will be pushed to move war to maintain trade supremacy.
The second strand of the literature, on the other hand, focuses on the impact of conflicts on trade. The works following this perspective are more homogeneous, and most authors agree to the disruptive effects on trade caused by wars. Barbieri and Levy (1999) analyse the impact of war on trade with adversary countries using seven dyads between 1870-1992, and they find that, although different across dyads, the general impact of conflict on trade is not particularly strong and mostly only temporary. Blomberg and Hess (2004) analyse more specifically the effect of all kind of conflicts, distinguishing between internal and external, and find that peace has a large and positive impact on trade. Anderton and Carter (2001) look at the effect of wars on global trade, and find that when major world power are at war significant pre and post war effects are observed, whereas impact is much smaller for conflicts between minor powers. Martin, Mayer and Thenig (2008) construct a theoretical model describing the likelihood of war and test it empirically; they find that likelihood of war is much smaller for countries involved in bilateral trade than for those involved in multilateral. Finally Glick and Taylor (2005) try to quantify the economic impact of the two world wars and claim that conflicts had negative effects on both belligerent and neutral countries with lags up to ten years. Altogether, the papers mentioned above do not always find coherent results, and such results were obtained from data from the last century only.The only exception is Rahman (2007) who uses British trade data from eighteen century, but concentrates manly on the impact of naval conflicts on trade. The majority of scholars (apart from Katherine and Levy) also finds long lasting effects of war; they claim commerce took several years before restoring its prewar level.

The effect of mercantilists wars on French trade does not fit this pattern. \cite{riley_seven_1986}, who concentrates on the case study of the Seven Years War, observes French trade series and he notices that there were no lags but on the contrary pre and post war loss compensation effects. This widely recognized fact about the effect of eighteenth century wars on French trade has led historians to research extensively the strategies of French merchants to cope with war. Neutral carriers were somewhat protected from British predation on the sea. When necessary, French merchants could even hide their cargo ownership behind a neutral partner. Or they could move to neutral countries and operate from there (\cite{marzagalli_was_2016}). Historians have even reflected that war periods might have been necessary to the functioning of the \textit{Éxclusif Colonial}, i.e. the theoretical monopoly of French merchants on French colonial trade (\cite{lespagnol_mondialisation_1997, morineau_vraie_1997, marzagalli_was_2016}). The argument rests on the large peace time trade imbalances between France and its Northern European clients for colonial goods that could have been balanced by large service income of Northern European merchants during war time as they, as neutrals, provided shipping and various trade services to the French empire. The quality of the available balance of payment data is not good enough to test that hypothesis.


The aim of this paper is to extend Riley’s work by analysing the available French data in the eighteenth century. So far the scholars have analysed the impact on trade of twentieth century wars and generalized the results. We believe that the effect of wars in twentieth century is different from that of other wars throughout history, and related data offer only a partial point of view. Thus, we are convinced that analysing less recent data is crucial to understand the general mechanisms relating trade and conflicts. We focus on the particular case of neutral countries and we look into the product breakdown of trade to observe the difference in impact between goods. we find indeed a general negative impact on trade, but looking at the product breakdown, the effect is much stronger in the case of colonial products, whereas in the case of European products the impact was even positive. In addition, we have also checked for the presence of war lags, as Glick and Taylor (2005) suggest. we find no evidence of war lag. On the contrary, we find a positive and significant coefficient for the two years following the war for all countries (around 40\%). Finally, we have tested for pre-war effects..  



\bibliographystyle{apa}
\bibliography{literature}




\end{document}



\begin{thebibliography}{9}

\bibitem{}
Reuven Glick, Alan M. Taylor 
\textit{Collateral damage: trade disruption and the economic impact of war}. 
The Review of Economics and Statistics, February 2010, 92(1): 102–127

\bibitem{}
J. C. Riley
\textit{The Seven Years War and the Old Regime in France}. 
Series: Princeton Legacy Library, 1986, Published by: Princeton University Press, Pages: 280

\bibitem{}
Ahmed S. Rahman
\textit{Fighting the Forces of Gravity - Seapower and Maritime Trade between the 18th and 20th Centuries}. 
Explorations in Economic History Volume 47, Issue 1, January 2010, Pages 28–48


\bibitem{}
Rafael Reuveny. 
\textit{Bilateral Import, Export, and Conflict/Cooperation Simultaneity}. 
International Studies Quarterly, Vol. 45, No. 1 (Mar., 2001), pp. 131-158

\bibitem{}
Guillaume Daudin. 
\textit{Domestic trade and market size in late eighteen century France}. 
The Journal of Economic History / Volume 70 / Issue	03 / September 2010, pp 716-743


\bibitem{}
S. Brock Blomberg, Gregory D. Hess
\textit{How much does violence tax trade?}. 
The Review of Economics and Statistics (Impact Factor:2.66). 11/2006; 88(4):599:612

\bibitem{}
Charles H. Anderton, John R. Carter
\textit{The Impact of War on Trade: An Interrupted Times-Series Study}. 
2001 Journal of Peace Research, vol. 38, no. 4, 2001, pp. 445–457

\bibitem{}
Loic Charles and Guillaume Daudin
\textit{Eighteen century international trade statistics sources and method}. 
Revue de l’OFCE

\bibitem{}
Katherine Barbieri and Jack S. Levy
\textit{Sleeping with the Enemy: The Impact of War on Trade}. 
Journal of Peace Research, Vol. 36, No. 4, Special Issue on Trade and Conflict (Jul., 1999), pp. 463-479

\bibitem{}
Jennings
\textit{Les marches du Nord dans le commerce francais au xvme siecle}. 
Rennes, Presses Universitaires de Rennes, 2006, 390 p


\bibitem{}
Ahmed S. Rahman, Darrell J. Glaser
\textit{Ex Tridenti Mercatus?- Sea-power and Maritime Trade in the Age of Globalization}. 
Journal of International Economics Volume 100, May 2016, Pages 95–111

\bibitem{}
Ahmed S. Rahman
\textit{Fighting the Forces of Gravity - Seapower and Maritime Trade between the 18th and 20th Centuries}. 
Explorations in Economic History Volume 47, Issue 1, January 2010, Pages 28–48

\bibitem{}
Scott L. Baier, Jeffrey H. Bergstrand
\textit{Do free trade agreements actually increase members' international trade?} 
Journal of International Economics 71 (2007) 72–95

\bibitem{}
Philip Martin, Thierry Mayer, Mathias Thoenig
\textit{Make Trade Not War?} 
Review of Economic Studies (2008) 75, 865–900

\bibitem{}
Mary Lindemann
\textit{The merchants republics: Amsterdam, Antwerp and Hamburg 1648-1790} 
Cambridge:	Cambridge University Press, 2015. 374 pp. ISBN 978-1-107-07443-9.

\bibitem{}
S. Brock Blomberg, Gregory D. Hess , and Siddarth Thacker
\textit{On the conflict-poverty nexus} 
ECONOMICS \&\ POLITICS 0954-1985 Volume 18 November 2006 No.3

\bibitem{}
Patrick Villiers
\textit{Marine Royale, corsaires et trafic dans l'Atlantique de Louis XIV à Louis XVI} 
Diffusion Septentrion, Press universitaires, Thèse à la carte

\bibitem{}
Francois Crouzet
\textit{La guèrre économique franco-anglaise au XVIII siècle} 
Paris, ed. Fayard, 2008 ISBN 978-2-213-63601-6

\bibitem{}
Charles Carrière
\textit{Negociants marseillais au XVIII siècle} 
Marseille, A. Robert, 1973. 2 vol. gr. in-8, 1.111 pages. (Institut historique de Provence.)

\bibitem{}
Pierrick Pourchasse
\textit{Le commerce du Nord. Les échanges commerciaux entre la France et l’Europe septentrionale au XVIIIe siècle, Rennes} Presses Universitaires de Rennes, 2006, 390 p., ISBN 978-2753500976.

\bibitem{}
Lagerqvist L. and E Nathorst-Boos, 
\textit{Mynt}, Bokforlaget PAN /Norstedts, Stockholm, 1968, p.78

\bibitem{}
Anne Husted Burleigh
\textit{John Adams} American presidents series, Transaction Publishers 2009, p. 189

\bibitem{}
Griffiths, David M. \textit{An American Contribution to the Armed Neutrality of 1780.} Russian Review 30, no. 2 (April 1971).

\end{thebibliography}

