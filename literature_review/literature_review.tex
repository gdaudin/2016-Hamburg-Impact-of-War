\documentclass[12pt,a4paper,titlepage,english]{article}
\usepackage[utf8]{inputenc}
\usepackage{amsmath}
\usepackage{amsfonts}
\usepackage{longtable}
\usepackage{amssymb}
\usepackage{graphicx}
\usepackage[a4paper, left=.6in,right=.6in,top=.8in,bottom=.8in,]{geometry}
\usepackage{tabularx,ragged2e,booktabs,caption}
\usepackage{setspace}
\setstretch{1.5}
\usepackage{tabularx, booktabs}
\usepackage{dcolumn} 
  \newcolumntype{d}[1]{D{.}{.}{#1}}    
\newcolumntype{Y}{>{\centering\arraybackslash}X}
\usepackage[T1]{fontenc}
\usepackage{babel}
\usepackage{natbib} 
\author{
  Elisa Tirindelli \\ Trinity College Dublin 
}
\title{Literature Review: \\
Relationship between trade and war}
\begin{document}

\maketitle
\section*{\cite{barbieri1999sleeping}}

There are two views on the impact of trade on wars: 
\begin{itemize}
\item{Liberal view: trade promotes peace}
\item{Realist view: trade has a negligible impact on conflicts or if it has any then it is a negative one (particularly asymmetric trade)}
\end{itemize}
There is however a common consent on the fact that war impacts trade negatively:
\begin{itemize}
\item{Liberal view: conflicts deters trade or adversely affects terms of trade.}
\item{Realist view: trade will terminate between adversaries because of relative gains concerns. }
\end{itemize}
Contrary to expectations however, there are several historical evidence of trading with the enemy. Neither of the two theories provides an explanation to why this happens. Barbieri and Levy try to conciliate this evidence with the two theories. In order to do, they use dyadic trade, between 1870 and 1992, for those seven countries whose data are available ten years before and after the war (no major power war data available). They use an interrupted time-series method. They observe that there is no major decline in "official" trade (even less so in smuggled trade) and that there is a ride in trade right after the end of the conflict. 

\section*{\cite{anderton2001impact}}
The "trade promotes peace" hypothesis (liberal view) rests upon three premises: (1) societies achieve salient economic gains from their trading relationships; (2) serious conflict among societies disrupts trade; and (3) premises 1 and 2 enter the calculus of political decisionmakers. If any one of these three premises does not hold, the liberal linkage between trade and peace is broken. \cite{barbieri1999sleeping} claim the second point fails to hold, so we cannot say trade promotes peace, however they only look at short wars between non-major powers. This paper, on the other hand, study the impact of war on trade of major power dyads, with the intention to fill the gap left by \cite{barbieri1999sleeping}. \textbf{The authors find that war between major powers is associated to a significant reduction in trade, and war between non major power also causes disruption, even though to a lesser extent.} They used an interrupted time series and estimate the following equation: 
\begin{equation*}
ln(trade)=\beta_0+\beta_1Trend+\beta_2WarLevel+\beta_3WarTrend+\beta_4PeaceLevel+\beta_5PeaceTrend+\epsilon
\end{equation*}
Where WarTrend is a counter of years scored 0 before war outbreak and then 1, 2, 3, . . . from the outbreak of war to the end of the time series (same for PeaceLevel).

\section*{\cite{blomberg2006much}}
The purpose of this paper is to calculate the economic cost of violence on trade and compare it to the economic cost of other trade barriers to see which is larger in magnitude. \textbf{The authors divide conflicts into four subcategories:} 
\begin{itemize}
\item{Terrorism (T)}
\item{External war (E)}
\item{Revolution (R)}
\item{Inter-ethnic fighting (IF)}
\end{itemize}
And analyse the effect of each single one of them on dyadic trade using a gravity model. In addition, they also analyse the effect of the aggregate effect of conflict on trade, using a factor analysis to create a synthetic measure of violence (TERIF). \\
They find that: first, conflict has a statistically significant and robust negative impact on bilateral trade flows (more than traditional tariff barriers). Second, different types of conflict have different negative impacts on trade. \\
Taken together, tariff equivalent cost of violence is between 8 and 19 percent. \textbf{This is higher than the costs from language and border and significantly higher than the benefits from GSPs and WTO/GATT membership.}


\section*{\cite{blomberg2006conflict}}
This paper constructs a \textbf{theoretical model} that relates conflicts to growth and growth to conflicts.
It does so by categorizing conflicts as either avoidable or unavoidable, by introducing leader's characteristics ($\gamma$ and $\rho$) and cost of conflicts ($\delta$). \textbf{Propositions: }
\begin{enumerate}
\item{If an economy has high returns to capital and a non selfish leader ($\rho\to0$), then cost of conflicts ($\delta$) are high enough to refrain leader to engage in avoidable conflicts. Hence conflicts hinders economic development but poor economic conditions do not generate conflicts. }
\item{If an economy has low returns to capital and a selfish leader ($\rho\to1$), then conflicts hinders economic development and poor economic conditions generate conflicts.}
\end{enumerate}
They test the model empirically using data from 152 countries between 1950 and 2000. They first estimate a reduced form equation model and then a simultaneous equations model, both to estimate growth and political instability.  They find disruptive effects of conflicts on economic outcomes on all economies, but only countries with initial lower levels of broad capital formation suffer get stuck in poverty-conflicts nexus. 


\section*{\cite{martin2008make}}
They derive theoretically the two-sided effect of trade on peace (positive for bilateral trade and negative for multilateral trade) and test empirically this prediction.
First they build a theoretical model where \textbf{disputes are exogenous but escalation to conflicts is endogenous}, and the timing of the game is as follows: a negotiation protocol is optimally chosen; then, information is privately revealed and negotiations take place. War occurs or not depending on the outcome of negotiations. \textbf{Then they incorporate this into a standard new trade theory model} (goods are imperfect substitutes) with trade costs . There are two implications in the model:
\begin{enumerate}
\item{An increase in bilateral imports of $i$ from $j$, as a ratio of country $i$'s
income, decreases the probability of escalation to war between these two countries.}
\item{An increase in multilateral imports (from countries other than $j$ ), as
a ratio of country $i$'s income, implies a higher probability of escalation to war between countries $i$ and $j$.}
\end{enumerate}
As a consequence globalization can increase the likelihood of small scale bilateral wars but decrease that of multilateral big scale wars.\\
They test the model with an \textbf{empirical analysis} on dyads between 1950 and 2000, using a simple gravity estimation and then one where all variables are in difference to the US. They test for negative prewar effects, but they are only significant and negative in the case relative to (substituting US with Switzerland yields insignificant results). Lags of 20 to 20 years are observed. They control with several other specifications. 
Ultimately, their empirical results confirm what the model says. 

\section*{\cite{rahman2010fighting}}
Rahman looks at English trade in 18th century and at a variety of trading partners between 19th and 20th century, to inspect the impact of \textbf{naval powers war on trade}. The novelty in this paper is that Rahman focuses on the \textbf{effect on trade of a conflict with a third party} (no endogeneity). In his estimate he evaluates both the effect of the conflict per se and extent by which the naval power can damage trade. He exploits a gravity model with panel data including the antagonistic naval power as an iceberg cost to trade. \\
He runs the empirical analysis splitting it into two periods:
\begin{enumerate}
\item{18th century (1710-1822): he analyses English trade with other countries, using English navy as
the main protagonistic fleet and the French navy as the main antagonistic fleet. As a measure of naval power he uses only the number of ships. In this specification, he is only interested in the intensive margin of antagonistic naval power. }
\item{19th and 20th century (1870-1947): he needs to define a more complex measure of naval power based on: relevant ships, changes in conflicts and changes in distances. In this setting he can disentangle the extensive margin of conflict (the effect of fighting a nation with global
reach) and the intensive margin of conflict (the effect of the extent of that global
power’s reach). He also uses an alternative measure of naval power in terms of capital investment in the navy.}
\end{enumerate}
He finds that naval forces, when summoned to antagonistic action, can inflict significant damage to international trade, particularly if those trading partners are located close to the naval power in question.

\section*{\cite{glaser2016ex}}
This paper attempts to assess the \textbf{ of sea power projection} by the major powers of the time on bilateral trade patterns from the early 1870s to the precipice of the Great War.
The authors develop a two-staged strategy: first they construct a model of naval power projection (simultaneous equation model, where naval deployment to a certain region at a certain time is jointly determined by all major naval powers) and then they incorporate it in a gravity model. 

\section*{\cite{glick2010collateral}}

\section*{\cite{baier2007free}}

\section*{\cite{o2005worldwide}}

\section*{\cite{daudin2010domestic}}

\section*{\cite{reuveny2001bilateral}}















\pagebreak

\renewcommand{\baselinestretch}{1.0}\normalsize

\renewcommand{\bibname}{\section{References}}

\bibliographystyle{apa}

\bibliography{literature}









\end{document}